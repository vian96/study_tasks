\documentclass[a4paper,12pt]{article} % добавить leqno в [] для нумерации слева
\usepackage[a4paper,top=1.3cm,bottom=2cm,left=1.5cm,right=1.5cm,marginparwidth=0.75cm]{geometry}
% % Работа с русским языком
\usepackage{cmap}% поиск в PDF
\usepackage{mathtext} % русские буквы в фомулах
\usepackage[T2A]{fontenc}% кодировка
\usepackage[utf8]{inputenc}% кодировка исходного текста
\usepackage[english,russian]{babel}% локализация и переносы

\usepackage{graphicx}
\usepackage{setspace}

\usepackage{wrapfig}
\usepackage{tabularx}

\usepackage{array}
\newcolumntype{P}[1]{>{\centering\arraybackslash}p{#1}}

\usepackage{hyperref}
\usepackage[rgb]{xcolor}
\hypersetup{
colorlinks=true,urlcolor=blue
}

% %  Дополнительная работа с математикой
\usepackage{amsmath,amsfonts,amssymb,amsthm,mathtools} % 0XF.FF9C264B28P-1031MS
\usepackage{icomma} % "Умная" запятая: $0,2$ --- число, $0, 2$ --- перечисление

% Номера формул
\mathtoolsset{showonlyrefs=true} % Показывать номера только у тех формул, на которые есть \eqref{} в тексте.

%  Шрифты
\usepackage{euscript} % Шрифт Евклид
\usepackage{mathrsfs} % Красивый матшрифт

%  Свои команды
\DeclareMathOperator{\sgn}{\mathop{sgn}}

%  Перенос знаков в формулах (по Львовскому)
\newcommand*{\hm}[1]{#1\nobreak\discretionary{}
{\hbox{$\mathsurround=0pt #1$}}{}}

\usepackage{mathtext}
\usepackage[T1,T2a]{fontenc}
\usepackage[utf8]{inputenc}
\usepackage[english, bulgarian, russian]{babel}

% % Заголовок
\author{Мартынов Иван Сергеевич}
\title{ Отчёт о выполнении лабораторной работы 1.4.1

Изучение физического маятника
}
\date{\today}

\begin{document}

\begin{titlepage}
\begin{center}
{\large МОСКОВСКИЙ ФИЗИКО-ТЕХНИЧЕСКИЙ ИНСТИТУТ (НАЦИОНАЛЬНЫЙ ИССЛЕДОВАТЕЛЬСКИЙ УНИВЕРСИТЕТ)}
\end{center}
\begin{center}
{\large Физтех-школа радиотехники и компьютерных технологий}
\end{center}


\vspace{4.5cm}
{\huge
\begin{center}
{\bf Отчёт о выполнении дифференцировании функций}
\end{center}
}
\vspace{2cm}
\begin{center}
{\LARGE Автор:\\ Мартынов Иван Сергеевич \\
\vspace{0.2cm}
Б01-110}
\end{center}
\end{titlepage}


\section{Введение}

Дифференцирование -- операция, обратная интегрированию. В данной работе мы будем ее проводить.

\section{Дифференцирование}

Дана следующая функция:\\ 
 \hspace{1cm}\\ 
 

$ f(x) =  \frac{ {x} + {2} }{  {\left( {x} ^ {{2}} + {2} \right)}  ^ {{0.5}} }  $\\ 
 \hspace{1cm}\\ 
 

Найдем производную самым очевидным рекурсивным образом:\\ 
 \hspace{1cm}\\ 
 

$ f'(x) =  \frac{  {\left( {1} + {0} \right)}  \cdot  {\left( {x} ^ {{2}} + {2} \right)}  ^ {{0.5}} - {0.5} \cdot  {\left( {x} ^ {{2}} + {2} \right)}  ^ { {\left( {0.5} - {1} \right)} } \cdot  {\left( {2} \cdot {x} ^ { {\left( {2} - {1} \right)} } \cdot {1} + {0} \right)}  \cdot  {\left( {x} + {2} \right)}  }{  {\left(  {\left( {x} ^ {{2}} + {2} \right)}  ^ {{0.5}} \right)}  ^ {{2}} }  $\\ 
 \hspace{1cm}\\ 
 

\section{Приведение ответа к адекватному виду}

Применив леммы (1.1.1) и (4.2.3), теорему (3.1.15) и используя определение (13.4.99), получим следующий упрощенный ответ:\\ 
 \hspace{1cm}\\ 
 

$ f'(x) =  \frac{  {\left( {x} ^ {{2}} + {2} \right)}  ^ {{0.5}} - {0.5} \cdot  {\left( {x} ^ {{2}} + {2} \right)}  ^ {{\left(-0.5\right)}} \cdot {x} \cdot {2} \cdot  {\left( {x} + {2} \right)}  }{ {x} ^ {{2}} + {2} }  $\\ 
 \hspace{1cm}\\ 
 

\section{Выводы}

В данной работе была проведена работа с проведениием дифференцирования и изучены способы упрощения математических выражений.

\end{document}
